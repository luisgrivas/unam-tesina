\documentclass[12pt]{article}
\usepackage{amsmath}
\usepackage{amsthm}
\usepackage{amssymb}
\usepackage{fancyhdr}
\usepackage[spanish]{babel}
%\usepackage{mathpazo,amsfonts}
\usepackage[margin=0.95in]{geometry}


\newcommand{\N}{\mathbb{N}}
\newcommand{\Z}{\mathbb{Z}}
\newcommand{\R}{\mathbb{R}}

\newtheorem{theorem}{Teorema}
\newtheorem{lemma}{Lema}


\title{Una Generalización de la Desigualdad Bonnesen-Grynkiewicz-Serra}
\author{LMP, LGR}
%%%% CONTENT %%%%%5
\begin{document}
\maketitle


La desigualdad de Bonnesen-Grynkiewicz-Serra (BGS) establece que %CITA

\begin{theorem} \label{GS}
Sean $A, B \subset \R^2$ conjuntos finitos y sea $L$ una línea en $\R^2$ que pasa por el origen. Entonces
$$\lvert A + B \rvert \geq \left( \frac{\lvert A \rvert }{\lvert p (A) \rvert} + \frac{\lvert B \rvert }{\lvert p (B) \rvert } - 1\right) \left( \lvert p(A) \rvert + \lvert p(B) \rvert - 1 \right),$$
donde $p: \R^2 \rightarrow L^\perp$ es la proyección ortogonal de $\R^2$ en $L^\perp$.
\end{theorem}

El teorema también es cierto para todas las dimensiones y para cualquier subespacio de $\R^d$. La idea central de la demostración de esta generalización es reducir la dimensión de los conjuntos mediante proyecciones en una dirección adecuada y utilizar el resultado anterior. Antes de proceder con el resultado necesitamos introducir algunos conceptos.

Sea $H$ un subespacio de $\R^d$. El mapeo $p: \R^d \mapsto H$ definido como $v \mapsto \sum_{i=1}^k (v \cdot w_i) \ w_i$  se le conoce como \textit{proyección ortogonal} de $\R^d$ en $H$. Geométricamente, si $v$ es un punto $\R^d$, $p(v)$ es el punto en $H$ de la recta que pasa por $v$ y es ortogonal a $H.$  

Sea $A$ un conjunto finito en $\R^d$. Denotemos por $L(A)$ el conjunto de líneas $l$ que pasan por el origen en $\R^d$ tales que 
$$\lvert (l + x) \cap A \rvert > 1,$$
para alguna $x \in \R^d$. En otras palabras, $L(A)$ es el conjunto de rectas que intersectan a $A$ en más de dos puntos. Note que $L(A)$ es finito, pues $A$ es finito y existe una única línea que pasa por dos puntos. Por finitud, escojamos una recta $H$ que pasa por el origen en $\R^d$ que no pertenece a $L(A)$. Si $q: \R^d \rightarrow H^\perp$ es la proyección ortogonal de $\R^d$ en $H^\perp$, entonces

$$\lvert q(A) \rvert = \lvert A \rvert.$$

Es decir, la proyección de $A$ en la dirección de $H$ conserva su cardinalidad. Queda claro que para generalizar el Teorema 1, solo queda encontrar una dirección que al momento de proyectar $A$ conserve el número de rectas paralelas que cubren a $A$.



Sea $L$ una recta que pasa por el origen en $\R^d$. Definamos a $C(A, L)$ al conjunto finito mínimo de líneas paralelas a $L$ que cubren a $A.$  

Recuerde que dos líneas paralelas distintas definen un único plano en $\R^d$. 

Definamos $P(A, L)$ al conjunto de planos que contienen a dos rectas para 

Let $C(A,L)$ be the minimum finite set of parallel lines to $L$ that covers $A$. Recall that any pair of parallel lines define a unique plane. Let $P(A, L)$ be the set of planes defined by every possible pair of lines in $C(A,L)$. Notice that $P(A, L)$ is finite, since $C(A,L)$ is finite. 

\begin{lemma}
    Let $A$ be a finite subset of $\R^d$ and let $L$ be a line through the origin in $\R^d$. Then there is a line $K$ through the origin in $\R^d$ such that
    \begin{itemize}
        \item[i)] $\lvert A \rvert  = \lvert p_{K^\perp}(A) \rvert$;
        \item[ii)] the number of lines parallel to $L$ that covers $A$ in $\R^d$ equals the number of lines parallel to $p_{K^\perp}(L)$ that covers $p_{K^\perp}(A)$ in $K^\perp.$
        
    \end{itemize}
\end{lemma}
\begin{proof}

Choose a line $K$ through the origin in $\R^d$ that is not $L$, it is not a member of $L(A)$, and it is not contained in any of the planes of $P(A,L)$  (we can do this because $L \cup L(A) \cup P(A,L)$ is finite). 
\begin{itemize}
\item[i)]  Suppose that there are $a, a^\prime \in A$ such that $p_{K^\perp}(a) = p_{k^\perp}(a^\prime).$ By Lemma 1, $a - a^\prime \in K.$ But this means that $a$ and $a^\perp$ belong to the same line parallel to $K$. But since $K$ is not a line in $L(A)$ and is not $L$, we have that $a = a^\prime.$ Therefore $\lvert A \rvert  = \lvert p_{K^\perp}(A) \rvert$.
\item[ii)] Now suppose that $p_{K^\perp}(x + L) = p_{K^\perp}( y + L)$ for $x + L, y + L$ in $C(A, L).$ Then $p_{K^\perp}(x-y) \in p_{K^\perp}(L)$. Since $K$ is not contained in any of the planes in $P(A, L)$, we have that $x + L = y + L.$ 
\end{itemize}
\end{proof}

Lemma 3 can be applied iteratively to any finite union of finite sets. That is the key to prove the next Theorem.

\begin{theorem}
    Theorem 1 is also true in dimension 3.
\end{theorem}
\begin{proof}
    Let 
    $$\Omega = A \cup B \cup (A + B) \cup p_{L^\perp}(A) \cup p_{L^\perp}(B) \cup p_{L^\perp}(A+B) \subset \R^3.$$
    By Lemma 3, we can choose a line $K$ through the origin in $\R^3$ such that
    $\lvert p_{K^\perp}(A) \rvert = \lvert A \rvert$, $\lvert p_{K^\perp}(B) \rvert = \lvert B \rvert$, $\lvert p_{K^\perp}(A + B) \rvert = \lvert A + B\rvert$, $\lvert p_{L^\perp}(p_{K^\perp}(A)) \rvert = \lvert p_{L^\perp}(A) \rvert$ and $\lvert p_{L^\perp}(p_{K^\perp}(B)) \rvert = \lvert p_{L^\perp}(B) \rvert$. Thus, by Theorem 1
    $$\lvert   p_{K^\perp}(A) +  p_{K^\perp}(B) \rvert \geq \left( \frac{\lvert  p_{K^\perp}(A) \rvert }{\lvert  p_{L^\perp}(p_{K^\perp}(A))  \rvert} + \frac{\lvert  p_{K^\perp}(B) \rvert }{\lvert  p_{L^\perp}(p_{K^\perp}(B))  \rvert } - 1\right) \left( \lvert p_{K^\perp}(A) \rvert + \lvert p_{K^\perp}(B) \rvert - 1 \right).$$ And by Lemma 2 we have,
     $$\lvert A + B \rvert \geq \left( \frac{\lvert A \rvert }{\lvert p_{L^\perp}(A) \rvert} + \frac{\lvert B \rvert }{\lvert p_{L^\perp} (B) \rvert } - 1\right) \left( \lvert p_{L^\perp}(A) \rvert + \lvert p_{L^\perp}(B) \rvert - 1 \right).$$
\end{proof}


\begin{theorem}
    Theorem 2 is also true if we replace the line $L$ by a plane $H$ in $\R^3.$
\end{theorem}
\begin{proof} (\textit{Sketch})
    Let $K$ be a line through the origin such that $K \subset H$ with the same properties as the proof of Theorem 1. Repeat the reasoning.  
\end{proof}





\end{document}