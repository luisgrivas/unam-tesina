\documentclass[12pt]{article}
\usepackage{amsmath}
\usepackage{amsthm}
\usepackage{amssymb}
\usepackage{fancyhdr}
%\usepackage{mathpazo,amsfonts}
\usepackage[margin=0.95in]{geometry}


\newcommand{\N}{\mathbb{N}}
\newcommand{\Z}{\mathbb{Z}}
\newcommand{\R}{\mathbb{R}}

\newtheorem{theorem}{Theorem}
\newtheorem{lemma}{Lemma}
\newtheorem{definition}{Definition}
\newtheorem{corollary}{Corollary}



\title{Notes on the Bonnesen-Grynkiewicz-Serra Inequality}
\author{LMP, LGR}


\begin{document}
\maketitle

The Bonnesen-Grynkiewicz-Serra Inequality states that %CITA
\begin{theorem} \label{GS}
Let $A, B \subset \R^2$ be finite sets, and let $L$ be a line in $\R^2$ through the origin. Then
$$\lvert A + B \rvert \geq \left( \frac{\lvert A \rvert }{\lvert p (A) \rvert} + \frac{\lvert B \rvert }{\lvert p (B) \rvert } - 1\right) \left( \lvert p(A) \rvert + \lvert p(B) \rvert - 1 \right).$$
Where $p: \R^2 \rightarrow L^\perp$ is the orthogonal projection of $\R^2$ on $L^\perp$.
\end{theorem}

The theorem is also true in all dimensions. In this note we will prove the case $d = 3.$ Let $H$ be a $k-$ dimensional subspace of $\R^d$ and let $\{w_1, \ldots w_k\}$ be a orthonormal base of $H.$ The map $p_H: \R^d \rightarrow H$ defined by $v \mapsto \sum_{i=1}^k (v \cdot w_i) \ w_i$ is called \textit{the orthogonal projection} of $\R^d$ on $H.$ The set $H^\perp = \{v \in \R^d: v \cdot w = 0, w \in H \}$ is called the \textit{orthogonal complement} of $H$. Direct computation shows that $H^\perp$ is a subspace of $\R^d$. Also, we have the following direct results. %%%%

\begin{lemma}
    The map $v \mapsto v - p_H (v)$ is the orthogonal projection of $\R^d$ on $H^\perp.$ The kernel of this map is $H$.
\end{lemma}

\begin{lemma}
    Let $A, B$ be subsets of $\R^d$. The map $p_H$ satisfies 
    $$p_H(A + B) = p_H(A) + p_H(B).$$
\end{lemma}
\begin{proof} Let $a \in A$ and $b\in B.$ Note that 
\begin{eqnarray*}
p_H(a + b) &=&  ((a + b)\cdot w_1) w_1 + ((a+b) \cdot w_2) w_2\\
&=& (a\cdot w_1) w_1 + (a \cdot w_2) w_2 +(b\cdot w_1) w_1 + (b \cdot w_2) w_2\\
&=& p_H(a) + p_H(b).
\end{eqnarray*}
The result follows.
\end{proof}

As a corollary of Lemma 2 we have
\begin{corollary}
A projection $p_{H}: R^d \rightarrow H$ maps parallel lines into parallel lines. 
\end{corollary}
\begin{proof}
    Let $L$ be a $1-$dimensional subspace of $\R^d$, and let $H$ be a subspace of $\R^d$ that is not $L.$ Then $p_H(L)$ is a subspace of $H$ of dimension $1$, that is, a line. By Lemma 2, $p_H(x + L) = p_H(x) + p_H(L)$ and $p_H(y + L) = p_H(y) + p_H(L)$. Therefore $p_H$ maps parallel lines into parallel lines. 
\end{proof}
Let $A$ be a finite subset of $\R^d$. Denote by $L(A)$ the set of lines $l$ through the origin in $\R^d$ such that
$$\lvert (l + x) \cap A \rvert > 1,$$
for some $x \in \R^d$. Notice that the set $L(A)$ is finite, since $A$ is finite and there is a unique line that passes through any pair of points. 

Let $C(A,L)$ be the minimum finite set of parallel lines to $L$ that covers $A$. Recall that any pair of parallel lines define a unique plane. Let $P(A, L)$ be the set of planes defined by every possible pair of lines in $C(A,L)$. Notice that $P(A, L)$ is finite, since $C(A,L)$ is finite. 

\begin{lemma}
    Let $A$ be a finite subset of $\R^d$ and let $L$ be a line through the origin in $\R^d$. Then there is a line $K$ through the origin in $\R^d$ such that
    \begin{itemize}
        \item[i)] $\lvert A \rvert  = \lvert p_{K^\perp}(A) \rvert$;
        \item[ii)] the number of lines parallel to $L$ that covers $A$ in $\R^d$ equals the number of lines parallel to $p_{K^\perp}(L)$ that covers $p_{K^\perp}(A)$ in $K^\perp.$
        
    \end{itemize}
\end{lemma}
\begin{proof}

Choose a line $K$ through the origin in $\R^d$ that is not $L$, it is not a member of $L(A)$, and it is not contained in any of the planes of $P(A,L)$  (we can do this because $L \cup L(A) \cup P(A,L)$ is finite). 
\begin{itemize}
\item[i)]  Suppose that there are $a, a^\prime \in A$ such that $p_{K^\perp}(a) = p_{k^\perp}(a^\prime).$ By Lemma 1, $a - a^\prime \in K.$ But this means that $a$ and $a^\perp$ belong to the same line parallel to $K$. But since $K$ is not a line in $L(A)$ and is not $L$, we have that $a = a^\prime.$ Therefore $\lvert A \rvert  = \lvert p_{K^\perp}(A) \rvert$.
\item[ii)] Now suppose that $p_{K^\perp}(x + L) = p_{K^\perp}( y + L)$ for $x + L, y + L$ in $C(A, L).$ Then $p_{K^\perp}(x-y) \in p_{K^\perp}(L)$. Since $K$ is not contained in any of the planes in $P(A, L)$, we have that $x + L = y + L.$ 
\end{itemize}
\end{proof}

Lemma 3 can be applied iteratively to any finite union of finite sets. That is the key to prove the next Theorem.

\begin{theorem}
    Theorem 1 is also true in dimension 3.
\end{theorem}
\begin{proof}
    Let 
    $$\Omega = A \cup B \cup (A + B) \cup p_{L^\perp}(A) \cup p_{L^\perp}(B) \cup p_{L^\perp}(A+B) \subset \R^3.$$
    By Lemma 3, we can choose a line $K$ through the origin in $\R^3$ such that
    $\lvert p_{K^\perp}(A) \rvert = \lvert A \rvert$, $\lvert p_{K^\perp}(B) \rvert = \lvert B \rvert$, $\lvert p_{K^\perp}(A + B) \rvert = \lvert A + B\rvert$, $\lvert p_{L^\perp}(p_{K^\perp}(A)) \rvert = \lvert p_{L^\perp}(A) \rvert$ and $\lvert p_{L^\perp}(p_{K^\perp}(B)) \rvert = \lvert p_{L^\perp}(B) \rvert$. Thus, by Theorem 1
    $$\lvert   p_{K^\perp}(A) +  p_{K^\perp}(B) \rvert \geq \left( \frac{\lvert  p_{K^\perp}(A) \rvert }{\lvert  p_{L^\perp}(p_{K^\perp}(A))  \rvert} + \frac{\lvert  p_{K^\perp}(B) \rvert }{\lvert  p_{L^\perp}(p_{K^\perp}(B))  \rvert } - 1\right) \left( \lvert p_{K^\perp}(A) \rvert + \lvert p_{K^\perp}(B) \rvert - 1 \right).$$ And by Lemma 2 we have,
     $$\lvert A + B \rvert \geq \left( \frac{\lvert A \rvert }{\lvert p_{L^\perp}(A) \rvert} + \frac{\lvert B \rvert }{\lvert p_{L^\perp} (B) \rvert } - 1\right) \left( \lvert p_{L^\perp}(A) \rvert + \lvert p_{L^\perp}(B) \rvert - 1 \right).$$
\end{proof}


\begin{theorem}
    Theorem 2 is also true if we replace the line $L$ by a plane $H$ in $\R^3.$
\end{theorem}
\begin{proof} (\textit{Sketch})
    Let $K$ be a line through the origin such that $K \subset H$ with the same properties as the proof of Theorem 1. Repeat the reasoning.  
\end{proof}


%Consider $A, B$ finite subsets of $\R^3.$ Define the set
%$$\Omega(A, B) = A \cup B \cup (A + B) \cup p_H(A) \cup p_H(B) \cup p_H(A+B) \subset \R^3.$$
%Since $\Omega(A,B)$ is finite, then there is a hyperplane $H^\prime$ in $\R^3$ such that 







\end{document}