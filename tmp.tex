Dados dos conjuntos $A$ y $B$ de un grupo abeliano $G$, la \textit{suma de Minkowski} de $A$ y $B$ se define como el conjunto 

$$ A + B = \{a + b: a \in A, b \in B\}.$$

\begin{teo}
	Sea $d \geq 2$ un entero y sea $0 < c < 2^d.$ Entonces existen constantes $k = k(c,d)$ y $s = s(c, d)$ tales que si $A \subset \R^d$ es finito, distinto del vacío que satisface que $\lvert A \rvert \geq k$ y $\lvert A + A \rvert < c \lvert A \rvert$, entonces $h_{d-1}(A, A) < s.$ 
\end{teo} %paper, grynkiewizc

\begin{teo}
	Sea $G$ un grupo abeliano libre de torsión y sean $A, B \subset G$ conjuntos finitos distintos del vacío. Entonces
	$$ \lvert A + B \rvert \geq \lvert A \rvert + \lvert B \rvert - 1$$. 
	La igualdad solo es posible si $A$ y $B$ son progresiones aritméticas con la misma diferencia o $\min\{\lvert A \rvert, \lvert B \rvert \} = 1.$
\end{teo} %pagina 26, grynkiewizc


\begin{teo}
	Sean $A, B \subset \R^2$ conjuntos finitos y distintos del vacío. Sea $l = \Rx$ una linea, sea $m$ el número de líneas paralelas a $l$ que intersectan a $A$ y sea $n$ el número de líneas paralelas a $l$ que intersectan a $B$. Entonces
	$$ \lvert A + B \rvert \geq \left( \frac{\lvert A \rvert}{m} + \frac{\lvert B \rvert}{n} - 1 \right)(m+n-1).$$
	
\end{teo} %pagina 80, grynkiewizc

\begin{lema}
	Si $a_1, \ldots, a_m, b_1, \ldots, b_n \subset \R$, donde $m, n \geq 1, entonces$

	$$\frac{1}{m+n-1}\sum_{r=2}^{m+n}\max\{a_i + b_j: i + j = 3, i \in [1, m], j \in [1,n]\}\geq \frac{1}{m}\sum_{i=1}^m a_i + \frac{1}{n}\sum_{i=1}^n b_i. $$
\end{lema}