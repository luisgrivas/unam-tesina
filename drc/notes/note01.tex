\documentclass[12pt]{article}
\usepackage[utf8]{inputenc}
\usepackage[spanish]{babel}
\usepackage{amsmath}
\usepackage{amsthm}
\usepackage{amssymb}
\usepackage{fancyhdr}
\usepackage{mathpazo,amsfonts}
\usepackage[margin=0.95in]{geometry}
\usepackage{tikz}


\usepackage[
backend=biber,
style=alphabetic,
sorting=ynt
]{biblatex}

% \addbibresource{../blb.bib} TODO: Verificar que la importación es correcta

\pagestyle{fancy}

\lhead{Nota 01}
\chead{Luis González}
\rhead{1 de febrero de 2025}

\newcommand{\N}{\mathbb{N}}
\newcommand{\Z}{\mathbb{Z}}
\newcommand{\Q}{\mathbb{Q}}
\newcommand{\R}{\mathbb{R}}
\newcommand{\E}{\mathbb{E}}

%\newtheorem{defi}[section]{Definición}
\newtheorem{prop}[section]{Proposición}
\newtheorem{lema}[section]{Lema}

\begin{document}
\section*{Aplicaciones Rápidas}
Estas notas se basan en lo presentado en %TODO: ingresar cita
Para un vértice $v$ en una gráfica $G$, definiremos al conjunto $N(v)$ como el
conjunto de vecinos de $v$; esto es, $N(v) := \{w \in V(G): (v, w) \in E(G)\}$.
Dado un conjunto de vértices $U$ de $G$, definimos la \textit{vecindad común}
$N(U)$ de $U$ como $N(U) := \bigcap_{u \in U} N(u)$.  
\begin{lema}[Lema básico]
    Sean $a, d, m, n, r$ enteros positivos. Sea $G = (V, E)$ una gráfica en
    $\vert V \vert = n$ vertices y grado promedio $d = 2 \vert E(G) /n$. Si existe un entero positivo $t$ tal que 
    $$\frac{d^t}{n^{t-1}} - \binom{n}{r} \left(\frac{m}{n}\right)^t \ge a,$$
    entonces $G$ contiene un subconjunto $U$ con al menos $a$ vertices tal que cada $r$ vertices en $U$ tiene al menos $m$ vecinos en comun.
\end{lema}
\begin{proof}
    Seleccione un conjunto $T$ de $t$ vertices en $V$ uniformemente al azar (así se escribe?) con repetición. Considere el conjunto $A = N(T)$ y 
    sea $X$ la cardinalidad de $A$. Por \textit{linealidad del valor esperado}, se tiene que
    $$\E[X] = \sum_{v \in V(G)} \left( \frac{\vert N(v) \vert}{n} \right)^t $$

\end{proof}
\end{document}
