\documentclass[12pt]{article}
\usepackage[utf8]{inputenc}
\usepackage[spanish]{babel}
\usepackage{amsmath}
\usepackage{amsthm}
\usepackage{amssymb}
\usepackage{fancyhdr}
% \usepackage{mathpazo,amsfonts}
\usepackage[margin=0.95in]{geometry}
\usepackage{tikz}


\usepackage[
backend=biber,
style=alphabetic,
sorting=ynt
]{biblatex}

% \addbibresource{../blb.bib} TODO: Verificar que la importación es correcta

\pagestyle{fancy}

\lhead{Luis González}
\chead{Dependent Random Choice}
\rhead{1 de febrero de 2025}

\newcommand{\N}{\mathbb{N}}
\newcommand{\Z}{\mathbb{Z}}
\newcommand{\Q}{\mathbb{Q}}
\newcommand{\R}{\mathbb{R}}
\newcommand{\E}{\mathbb{E}}

%\newtheorem{defi}[section]{Definición}
\newtheorem{prop}[section]{Proposición}
\newtheorem{lema}[section]{Lema}

\begin{document}
\section*{Versiones de Selección Dependiente Aleatoria}

Para un vértice $v$ en una gráfica $G$, definiremos al conjunto $N(v)$ como el
conjunto de vecinos de $v$; esto es, $N(v) := \{w \in V(G): (v, w) \in E(G)\}$.
Dado un conjunto de vértices $U$ de $G$, definimos la \textit{vecindad común}
$N(U)$ de $U$ como $N(U) := \bigcap_{u \in U} N(u)$.  

\begin{lema}
    Sea $t$ un entero y sea $G$ una gráfica en $n$ vértices con grado promedio
    $d = \epsilon n$. Entonces $G$ contiene una $t$-tupla de vértices que tiene 
    al menos $\epsilon^t n - \binom{t}{2}$ vecinos en común.
\end{lema}

\begin{proof}
    Sean $u_1, \ldots, u_t$ vértices de $G$ escogidos uniformemente con reemplazo y sea $T = \{u_1, \ldots u_t\}$.
    Dado un vértice $v$ en $G$, ¿cuál es la probabilidad de que este sea un vecino común de 
    $T$? Por definición, $v \in N(T)$ si y solo si $(v, u_i) \in E(G)$ para todo $u_i \in T$.
    En otras palabras, $v \in N(T)$ si y solo si $T \subset N(v)$. Esto ocurre con probabilidad
    $\left(\frac{d(v)}{n}\right)^t$.

    Ahora bien, queremos estimar el valor esperado del tamaño de $N(T)$. Por linealidad, obtenemos
    \begin{eqnarray*}
        \E[\vert N(T) \vert] &=& \sum_{v \in V(G)} Pr(v \in N(T))  \\
                             &=& \sum_{v \in V(G)} Pr(T \subset N(v)) \\
                             &=& \sum_{v \in V(G)} \left(\frac{d(v)}{n}\right)^t \\
                             &=&  n^{1-t} \cdot \frac{\sum_{v \in V(G)} d(v)^t}{n} \\
                             &\geq& n^{1-t} \cdot \left( \frac{1}{n} \cdot \sum_{v \in V(G)} d(v) \right)^t\\
                             &=& n^{1-t} d^t \\ 
                            &=& \epsilon^t n,
        \end{eqnarray*}
    donde la desigualdad se debe a la convexidad de la función $x \mapsto x^t$. Por tanto, existe un conjunto de 
    vértices $T$ que tiene al menos $\epsilon^t n$ vecinos en común.

    Con lo realizado hasta ahora pareciera que ya hemos acabado (obteniendo incluso un resultado más fuerte). Desafortunadamente
    cuando seleccionamos a los vértices $u_i$, existe la posibilidad de que estos no sean todos distintos. ¿Cuál es la probabilidad
    de que esto ocurra? Note que, para $i \neq j$, se tiene que $Pr(u_i = u_j) = \frac{1}{n}$. 
    Por lo que $Pr(\vert T \vert < t) \leq \frac{1}{n}\cdot \binom{t}{2}$. Luego,
    \begin{eqnarray*}
    \E[\vert N(T)\vert] &=& \E[\vert N(T)\vert | \vert T \vert = t] \cdot Pr( \vert T \vert = t) + \E[\vert N(T)\vert | \vert T \vert < t] \cdot Pr( \vert T \vert < t) \\
                        &\leq& \E[\vert N(T)\vert | \vert T \vert = t] \cdot Pr( \vert T \vert = t) + n \cdot \frac{1}{n}\cdot \binom{t}{2}\\
                        &=& \E[\vert N(T)\vert | \vert T \vert = t] \cdot Pr( \vert T \vert = t) + \binom{t}{2}.
    \end{eqnarray*}
    Por tanto,
    $$\E[\vert N(T)\vert | \vert T \vert = t] \cdot Pr( \vert T \vert = t) \geq \epsilon^t n - \binom{t}{2}.$$
\end{proof}


\begin{lema}[Lema básico]
    Sean $a, d, m, n, r$ enteros positivos. Sea $G = (V, E)$ una gráfica en
    $\vert V \vert = n$ vertices y grado promedio $d = 2 \vert E(G) /n$. Si existe un entero positivo $t$ tal que 
    $$\frac{d^t}{n^{t-1}} - \binom{n}{r} \left(\frac{m}{n}\right)^t \ge a,$$
    entonces $G$ contiene un subconjunto $U$ con al menos $a$ vertices tal que cada $r$ vertices en $U$ tiene al menos $m$ vecinos en comun.
\end{lema}
\begin{proof}
    Seleccione un conjunto $T$ de $t$ vertices en $V$ uniformemente al azar (así se escribe?) con repetición. Considere el conjunto $A = N(T)$ y 
    sea $X$ la cardinalidad de $A$. Por \textit{linealidad del valor esperado}, se tiene que
    $$\E[X] = \sum_{v \in V(G)} \left( \frac{\vert N(v) \vert}{n} \right)^t $$

\end{proof}
\end{document}
