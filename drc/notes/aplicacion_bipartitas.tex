\documentclass[14pt]{article}
\usepackage[utf8]{inputenc}
\usepackage[spanish]{babel}
\usepackage{amsmath}
\usepackage{amsthm}
\usepackage{amssymb}
\usepackage{fancyhdr}
\usepackage[margin=0.95in]{geometry}
\usepackage{tikz}

\usepackage[
  backend=biber,
  style=alphabetic,
  sorting=ynt
]{biblatex}

\addbibresource{../blb.bib}

\pagestyle{fancy}

\lhead{Luis González}
\chead{Dependent Random Choice}
\rhead{11 de abril de 2025}

\newcommand{\N}{\mathbb{N}}
\newcommand{\Z}{\mathbb{Z}}
\newcommand{\Q}{\mathbb{Q}}
\newcommand{\R}{\mathbb{R}}
\newcommand{\E}{\mathbb{E}}

%\newtheorem{defi}[section]{Definición}
\newtheorem{prop}[section]{Proposición}
\newtheorem{lema}[section]{Lema}
\newtheorem{theorem}{Teorema}

\begin{document}
\section*{Aplicación de SDA a Gráficas Bipartitas}

Dada una gráfica $H$, el \textit{número de Turán} o \textit{extremal}
$ex(n, H)$, representa el número máximo de aristas de una gráfica en
$n$ vértices que no contiene una copia de $H$. En \cite{fox2010drc}
se presenta el siguiente teorema:

\begin{theorem}
  Si $H = (A \cup B, F)$ es una gráfica bipartita en la que todos los
  vértices de $B$ tienen grado
  a lo más $r$, entonces $ex(n, H) \leq c n^{2 - 1 / r}$, donde $c =
  c(H)$ es una constante que solo depende de $H$.
\end{theorem}

La demostración presentada en este trabajo es una aplicación directa
de \textit{Selección Dependiente Aleatoria} (SDA)
\begin{theorem}(Lema Básico)\label{drc}
  Sean $\alpha, m, n, r, u$ enteros positivos. Sea $G$ una gráfica en
  $n$ vértices y al menos $ \frac{\alpha n^2}{2}$ aristas. Si existe
  entero positivo $t$
  tal que
  $$n \alpha^t - \binom{n}{r} \left(\frac{m}{n}\right)^t \geq u$$
  entonces $G$ contiene un conjunto $U \subset V(G)$ de vértices tal que
  $\vert U \vert \geq u$ y tal que todo subconjunto $S$ de $U$ con
  $\vert S \vert = r$
  tiene al menos $m$ vecinos en común.
\end{theorem}
% TODO: hay qu emejorar esto
Demostración usando DRC. Sea $G$ una gráfica con $e(G) > c n^{2 -1 /
r}$. Sean $\alpha = (2c n^{-1/r})$,
$\vert A \vert = a$, $\vert B \vert = b$, $m = a + b$, $t = r$
y $c = \max(a^{1/r}, \frac{3(a + b)}{r})$. Entonces

\begin{eqnarray*}
  n \alpha^t - \binom{n}{r} \left(\frac{m}{n}\right)^t &=& n (2c
  n^{-1/r})^r - \binom{n}{r} \left(\frac{a + b}{n}\right)^r\\
  &\geq& (2c)^r - \frac{n^r}{r!} \left((\frac{a + b}{n}\right)^r \\
    &\geq& (2c)^r - \left(\frac{e (a + b)}{r}\right)^r\\
    &\geq& c^r \geq a
  \end{eqnarray*}
  Por tanto, el Teorema \ref{drc} establece que existe un subconjunto
  $U$ de $V(G)$ tal que $\vert U \vert = a$ y en el que todos sus
  subconjuntos de tamaño
  $r$ tienen al menos $a + b$ vecinos en común.

  Sea $f: A \rightarrow U$ una función inyectiva cualquiera.
  Demostraremos que $f$ se puede extender a $B$ de tal manera que
  esta extensión es un encaje de $H$ en $G$.

  El siguiente texto presenta notas sobre el trabajo realizado en
  \cite{furedi1991}.

  \end{document}
