\section{Números de Turán para Gráficas Bipartitas}

\begin{theorem}\label{turan-drc}
  Si $H = (A \cup B, F)$ es una gráfica bipartita en la que todos los
  vértices de $B$ tienen grado
  a lo más $r$, entonces $ex(n, H) \leq c n^{2 - 1 / r}$, donde $c =
  c(H)$ es una constante que solo depende de $H$.
\end{theorem}
\begin{proof}
  Sean $\vert A \vert = a$, $\vert B
  \vert = b$, $m = a + b$, $t = r$
  y $c = \max(a^{1/r}, \frac{3(a + b)}{r})$. Sea $G$ una gráfica con
  $e(G) > 2 c n^{2 -1 /
  r}$. De manera que $d = 2e(G) / n > 2 c n^{1 - 1/r}$. Entonces

  \begin{eqnarray*}
    \frac{d^{t}}{n^{t-1}} - \binom{n}{r} \left(\frac{m}{n}\right)^t
    &=& \frac{(2c
    n^{1-1/r})^r}{n^{r-1}}  - \binom{n}{r} \left(\frac{a + b}{n}\right)^r\\
    &\geq& (2c)^r - \frac{n^r}{r!} \left((\frac{a + b}{n}\right)^r \\
      &\geq& (2c)^r - \left(\frac{e (a + b)}{r}\right)^r\\
      &\geq& c^r \geq a
    \end{eqnarray*}
    Por tanto, el Teorema \ref{drc} establece que existe un subconjunto
    $U$ de $V(G)$ tal que $\vert U \vert = a$ y en el que todos sus
    subconjuntos de tamaño
    $r$ tienen al menos $a + b$ vecinos en común.

    Sea $f: A \rightarrow U$ una función inyectiva cualquiera.
    Demostraremos que $f$ se puede extender a $B$ de tal manera que
    esta extensión es un encaje de $H$ en $G$. Etiquete los vértices de $B$ como
    $v_1, \ldots, v_b$. Suponga que el vértice por encajar es $v_i \in
    B$. Sea $N_i \subset A$ la vecindad de $v_i$ en $H$; por lo que
    $\vert N_i \vert \leq r$. Dado que $f(N_i)$ es un subconjunto de
    $U$ de cardinalidad a lo más $r$, existen al menos $a + b$ vértices
    adyacentes a todos los vértices en $f(N_i)$. Como el total de
    vértices por encajar es $a + b$, existe un vértice $v \in V(G)$ que
    no se ha usado en el encaje y que es adyacente a $f(N_i)$ en $G$.
    Haga $f(v_i)  = v$. Se observa que este procedimiento puede
    continuar hasta terminar de definir $f(v_i)$ para todo $i = 1,
    \ldots, b$. Por tanto $f$ es un encaje de  $H$ en $G$.
  \end{proof}

  \section{Números de Ramsey para Cubos}
  Para una gráfica $H$, el \textit{número de Ramsey} $r(H)$, es el
  mínimo entero positivo $N$ tal que cualquier \textit{2-coloración}
  de las aristas de la gráfica completa en $N$ vértices contiene una
  copia monocrómatica de $H$.
  \begin{theorem}
    $r(Q_r) \leq 2^{3r}$.
  \end{theorem}
  \begin{proof}
    En cualquier 2-coloración de las aristas de una gráfica comleta
    en $N = 2^{3r}$ vértices, el conjunto más denso entre los dos
    colores tiene al menos
    $\frac{1}{2} \binom{N}{2} \geq 2^{-7/3} N^2$ aristas. Sea $G$ la gráfica
    del color más denso. De manera que el grado promedio $d$ de $G$
    es al menos $2^{-4/3}N$. Aplicando el Teorema \ref{drc} con $t =
    \frac{3}{2}r$, $m = 2^r$ y $a = 2^{r-1}$, tenemos que
    $$\frac{d^t}{N^{t-1}} - \binom{N}{r} \left(\frac{m}{N}\right)^t
    \geq 2^{-4/3 t} N - N^{r-t}m^t / r! \geq 2^r - 1 \geq 2^{r-1}. $$
    Por tanto, existe un subconjunto $U$ de $G$ de tamaño $2^{r-1}$
    en el que todo $r$ vértices en $U$ tiene al menos $2^r$ vecinos en común.
    Dado que $Q_r$ es una gráfica bipartita $r$-regular con partes de
    tamaño $2^{r-1}$, podemos encontrar un encaje de $Q_r$ en $G$ tal
    y como lo hicimos en el Teorema \ref{turan-drc}
  \end{proof}
  % \section{Un Problema del tipo Turán-Ramsey para Gráficas libres de $K_4$}
