\chapter{Teoría Extremal}

\section{La Subgráfica Prohibida}
Dada una gráfica $H$, el \textit{número de Turán} o \textit{extremal}
$ex(n, H)$, representa el número máximo de aristas de una gráfica en
$n$ vértices que no contiene a $H$, \textit{subgráfica prohibida},
como subgráfica.

\section{Teorema de Turán}
Una gráfica \textit{$k-$ partita} es aquella cuyo conjunto de
vértices puede ser particionado en $k$ subconjuntos, o
\textit{partes}, de tal manera que ninguna arista tiene como extremos
a vértices de una misma parte. Una gráfica $k-$partita es
\textit{completa} si cualesquiera dos vértices en diferentes partes
son adyacentes. Una gráfica completa $k$-partita en $n$ vértices
cuyas partes tienen tamaños que difieren a lo más en un vértice se
les conoce como  \textit{gráficas de Turán} y se denotan como $T_{k, n}$.
\begin{theorem} \label{turan}
  Sea $G$ una gráfica simple que no contiene a $K_k$, para $k \geq
  2$. Entonces $e(G) \leq e(T_{{k-1}, n})$, con igualdad si y solo si
  $G \simeq T_{{k-1}, n}$.
\end{theorem}

% TODO: agregar cita de Zykov (1949)
% TODO: no funcionan las Lambdas.corregir
% NOTE: demostración del Bondy-Murty
\begin{proof}
  Procederemos por inducción sobre $k$. El teorema se satisface
  trivialmente para $k=2$.  Suponga que se satisface para cualquier
  entero menor que $k$, y sea $G$ una gráfica simple que no contiene
  a $K_k$. Seleccione un vértice $v$ con grado máximo $\Lambda$ en
  $G$. Haga $A = N(v)$ y $B = V \setminus A$. Entonces
  $$e(G) = e(A) + e(A, B) + e(B).$$
  Dado que $G$ no contiene a ningún $K_k$, la gráfica inducida $G[A]$
  no contiene a $K_{k-1}$. Por tanto, por inducción,
  $$e(A) \leq e(T_{k-2, \Lambda}),$$
  con igualdad si y solo si $G[A] \simeq T_{{k-2}, \Lambda}$. Además,
  dado que toda arista de $G$ incidente a un vértice de $B$ pertence
  a $E[A, B]$ o $E( B )$,
  $$e(A, B) + e(B) \leq \Lambda (n - \Lambda),$$
  con igualdad si y solo si $B$ es un conjunto independiente y cuyos
  miembros tienen grado $\Lambda$. Luego, $e(G) \leq e(H)$, donde $H$
  es una gráfica obtenida de una copia de $T_{{k-2}, \Lambda}$ y
  añadiendo una copia de un conjunto independiente de $n-\Lambda$
  vértices y uniendo cada uno de estos vértices con cada vértice de
  $T_{{k-2}, \Lambda}$. Observe que $H$ es una gráfica completa
  $(k-1)$-partita en $n$ vértices. Entonces $e(H) \leq e(T_{{k-1},
  n})$ con igualdad si y solo si $H \simeq T_{{k-1}, n}$. Se sigue
  que $e(G) \leq e(T_{k-1, n})$, con igualdad si y solo si $G \simeq
  T_{k-1, n}$.
\end{proof}
% TODO: agregar demostracion usando probabilistic method

\section{Teorema de Erdős-Stone}

\section{Números de Ramsey}
Un \textit{clique} de una gráfica es un conjunto de vértices todos de ellos adyacentes.
