% \chapter{Teoría Extremal}

\section{La Subgráfica Prohibida}

\begin{definition}[Subgráficas Prohibidas] Sea $\mathcal F = \{F_1,
  \ldots, F_k\}$ una familia de gráficas de orden a lo más $n$. Escribimos
  $ex(n, \mathcal F) = ex(n, F_1, \ldots, F_k)$ para el máximo número de aristas
  en una gráfica de orden $n$ que no tiene a ningún $F_i$ como subgráfica.
\end{definition}

\begin{definition}[Número de Turán]
  Dada una gráfica $H$, el \textit{número de Turán} o \textit{extremal}
  $ex(n, F)$, representa el número máximo de aristas de una gráfica en
  $n$ vértices que no contiene a $F$, \textit{subgráfica prohibida},
  como subgráfica.
\end{definition}

Es de nuestro interés estudiar el comportamiento asintótico de $ex(n,
\mathcal F)$ cuando $n \to \infty$.

\subsection{Teorema de Turán}

\begin{theorem}[Mantel]
  Sea $G$ una gráfica de orden $n$ y \textit{libre de triángulos}.
  Esto es, $G$ no
  contiene un clique de orden tres. Entonces $e(G) \leq \frac{n^2}{4}$.
\end{theorem}

\begin{theorem}
  Sea $G$ una gráfica $k$-partita con partes de tamaño $a_1, \ldots,
  a_k$. Entonces
  $e(G) \le \frac{1}{2} \sum_{i=1}^k a_i (n - a_i)$.
\end{theorem}

\begin{corollary}
  La gráfica de Turán $T_{k, n}$ es la gráfica $k$-partita completa
  de orden $n$ con más aristas entre todas las gráficas $k$-partitas
  completas de ordern $n$.
\end{corollary}

\begin{theorem}[Turán] \label{turan}
  Sea $G$ una gráfica que no contiene a $K_k$, para $k \geq
  2$. Entonces $e(G) \leq e(T_{{k-1}, n})$, con igualdad si y solo si
  $G \simeq T_{{k-1}, n}$.
\end{theorem}

% TODO: agregar cita de Zykov (1949)
% NOTE: demostración del Bondy-Murty
\begin{proof}
  Procederemos por inducción sobre $k$. El teorema se satisface
  trivialmente para $k=2$.  Suponga que se satisface para cualquier
  entero menor que $k$, y sea $G$ una gráfica simple que no contiene
  a $K_k$. Seleccione un vértice $v$ con grado máximo $\Delta$ en
  $G$. Haga $A = N(v)$ y $B = V \setminus A$. Entonces
  $$e(G) = e(A) + e(A, B) + e(B).$$
  Dado que $G$ no contiene a ningún $K_k$, la gráfica inducida $G[A]$
  no contiene a $K_{k-1}$. Por tanto, por inducción,
  $$e(A) \leq e(T_{k-2, \Delta}),$$
  con igualdad si y solo si $G[A] \simeq T_{{k-2}, \Delta}$. Además,
  dado que toda arista de $G$ incidente a un vértice de $B$ pertence
  a $E[A, B]$ o $E( B )$,
  $$e(A, B) + e(B) \leq \Delta (n - \Delta),$$
  con igualdad si y solo si $B$ es un conjunto independiente y cuyos
  miembros tienen grado $\Delta$. Luego, $e(G) \leq e(H)$, donde $H$
  es una gráfica obtenida de una copia de $T_{{k-2}, \Delta}$ y
  añadiendo una copia de un conjunto independiente de $n-\Delta$
  vértices y uniendo cada uno de estos vértices con cada vértice de
  $T_{{k-2}, \Delta}$. Observe que $H$ es una gráfica completa
  $(k-1)$-partita en $n$ vértices. Entonces $e(H) \leq e(T_{{k-1},
  n})$ con igualdad si y solo si $H \simeq T_{{k-1}, n}$. Se sigue
  que $e(G) \leq e(T_{k-1, n})$, con igualdad si y solo si $G \simeq
  T_{k-1, n}$.
\end{proof}
% TODO: agregar demostracion usando probabilistic method

Sea $G$ una gráfica. Recuerde que un conjunto independiente $S$ de
vértices en $G$ es aquel en el que ninguna arista en $G$ tiene como
extremos dos vértices en $S$. De manera complementaria, un clique $C$
en $G$ es un conjunto de vértices en $G$ en el cual cualquier par de
vértices es adyacente. Es decir, un clique $C$ con $\vert C \vert =
k$ induce una gráfica isomorfa a $K_k$, $G[C] \equiv K_k$. Por tanto
el Teorema de Turán puede reescribirse como \textit{toda gráfica $G$
  en $n$ vértices y con más de $e(T_{k-1, n})$ aristas contiene un
clique de orden $k$}.

Los conjuntos independientes y los cliques están relacionados de la siguiente
manera. Para una gráfica $G$, sea $\alpha(G)$ la cardinalidad del conjunto
independiente de vértices más grande en $G$. Y sea $\omega(G)$ la
cardinaliad del
clique más grande en $G$. Entonces
$$\omega(G) = \alpha(\bar{G}),$$
donde $\bar{G}$ es la gráfica complemento de $G$. Por tanto, toda
proposición cierta para cliques puede reescribirse en términos de
conjuntos independientes. Tal es el caso del Teorema de Turán. El
siguiente resultado fue demostrado por
Caro \cite{caro1979} y Wei  %TODO: agregar la cita de wei

\begin{theorem}[Caro y Wei]
  Sea $G = (V, E)$ una gráfica. Entonces $$ \alpha(G) \geq \sum_{v
  \in V} \frac{1}{d(v) + 1}.$$
\end{theorem}

%TODO: cambiar simbolo de orden
\begin{proof}
  Sea $<$ una orden total de $V$ elegido uniformemente. Defina
  $$I = \{v \in V: vw \in E \Rightarrow v < w \}.$$
  Sea $X_{v}$ la variable aleatoria indicadora para $v \in I$ y $X =
  \sum_{v \in V} X_v = \vert I \vert$. Para cada $v$,
  $$E[X_v] = Pr[v \in I] = \frac{1}{d(v) + 1},$$
  dado que $v \in I$ si y solo si $v$ es el elemento menor entre $v$
  y sus vecinos. Luego
  $$ E[X] = \sum_{v \in V} \frac{1}{d(v) + 1}. $$
  Por tanto existe un orden total $<$ con
  $$\vert I \vert \ge \sum_{v \in V} \frac{1}{d(v)  +1 }.$$
  Pero si $x, y \in I$ y $xy \in E$ entonces $x < y$ y $y < x$, una
  contradicción. Por tanto $I$ es un conjunto independiente y
  $\alpha(G) \ge \vert I \vert.$
\end{proof}

% TODO: revisar el statement del teorema
\begin{theorem} (Turán)
  Sea $G$ una gráfica en $n$ vértices y $e(G) \le e(T_{k-1, n})$
  aristas. Entonces $\alpha(G) \geq k$ con $\alpha(G) = k$ si y solo
  si $G \equiv T_{k-1, n}$.
\end{theorem}

\begin{proof} % TODO: completar
  La gráfica $T_{k-1, n}$ tiene $\sum_{v \in V}(d(v) + 1)^{-1} = $
\end{proof}
\subsection{Teorema de Erdős-Stone} %TODO: no se si esto lo vaya a incluir
\begin{theorem}
  Sea $G$ una gráfica. Entonces
  $$ex(n, G) = \left( 1 - \frac{1}{\chi(G) - 1} + o(1) \right)\frac{n^2}{2}. $$
\end{theorem}

\section{Números de Ramsey}
Un \textit{clique} de una gráfica es un conjunto de vértices todos de
ellos adyacentes.
