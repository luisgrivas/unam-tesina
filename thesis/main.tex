\documentclass{article}
\usepackage[spanish,mexico]{babel}
\usepackage[margin=2.5cm]{geometry}
\usepackage{graphicx}
\usepackage{amsmath}
\usepackage{amsthm}
\usepackage{amssymb}

\newcommand{\N}{\mathbb{N}}
\newcommand{\Z}{\mathbb{Z}}
\newcommand{\Q}{\mathbb{Q}}
\newcommand{\R}{\mathbb{R}}
\newcommand{\E}{\mathbb{E}}

\newtheorem{theorem}{Teorema}[section]
\newtheorem{corollary}[theorem]{Corolario}
\newtheorem{lemma}[theorem]{Lema}
\newtheorem{proposition}[theorem]{Proposición}
\newtheorem*{axiom}{Axioma}

\theoremstyle{definition}
\newtheorem{definition}{Definición}[section]

\title{Selección Dependiente Aleatoria}
\author{Luis Felipe González Rivas}

\begin{document}
\maketitle
% \thispagestyle{empty}
\centering
\includegraphics[scale=0.5]{logo-unam.jpg}

\begin{center}
    \LARGE \textbf{UNIVERSIDAD NACIONAL AUTÓNOMA DE MÉXICO}
    \vspace{0.2cm}
   
    \Large PROGRAMA DE MAESTRÍA Y DOCTORADO EN CIENCIAS MATEMÁTICAS Y DE LA ESPECIALIZACIÓN EN ESTADÍSTICA APLICADA
    \vspace{3.0cm}

    \Large \textit{TITULO}
    \vspace{1.2cm}

    \Large TESINA QUE PARA OPTAR POR EL GRADO DE:
    
    MAESTRO (A) EN CIENCIAS
    \vspace{1.2cm}

    \Large PRESENTA:
    
    LUIS FELIPE GONZALEZ RIVAS
    \vspace{1.5cm}

    \Large ADRIANA HANSBERG PASTOR
    
    INSTITUTO DE MATEMATICAS  % CREO (INDICAR ENTIDAD DE ADSCRIPCIÓN, INDEPENDIENTEMENTE DE QUE ÉSTA PERTENEZCA A LA UNAM O NO)
    \vspace{1.5cm}

    \Large LUGAR, MES Y AÑO EN QUE SE REALIZÓ EL REGISTRO
\end{center}
\newpage

% \chapter{Introducción}

\section{Preliminares}
\begin{definition}
  Una gráfica $G$ es una pareja ordenada $(V, E)$, donde $V$
  es un conjunto cuyos elementos llamamos \textit{vértices} y $E$ es
  un conjunto cuyos elementos son parejas de vértices y que llamamos
  \textit{aristas}.
  %TODO: ver libros de olimpiadas para ver las def en español
\end{definition}

Si $e$ es una arista y $u$ y $v$ son vértices tal que $\psi_G(e)
= \{u, v\}$, entonces se dice que $e$ \textit{une} a los vértices
$u$ y $v$, y a estos vértices se les conoce como \textit{extremos} de $e$.
El grado de un vértice $v$, denotado como $d(v)$, es el número de
aristas en $G$ que inciden en $v$.

\begin{definition}
  Sea $v$ un vértice en una gráfica $G$. Definimos a la vecindad
  $N(v)$ de $v$ como el conjunto de todos los vértices adyacentes a
  $v$ en $G$. De
  manera similar, sSi $U$ es un conjunto, definimos a la vencidad de $U$ como
  el conjunto de vértices en $G$ adyacentes a todo vértice de $U$.
\end{definition}

\begin{definition}
  El grado $d(v)$ de un vértice $v$ en una gráfica $G$ es la
  cardinalidad de la vecindad de $v$, y escribimos
  $d(v) = \vert N(v) \vert$. Al grado máximo en una gráfica $G$ lo
  denotamos como $\Delta$. El \textit{grado promedio} en una gráfica
  se define como $d = \sum_{v \in V} d(v) / n$.
\end{definition}

\begin{theorem}[del Saludo] Para cualquier gráfica $G$ con $m$
  aristas se tiene que
  $$\sum_{v \in V(G)} d(v) = 2m.$$
\end{theorem}
%TODO: proof
El \textit{grado promedio} $\bar{d}$ de una gráfica $G$ en $n$
vértices es $\sum_{ v
\in V(G)} d(v) / n$. Así pues, el Teorema anterior implica que

\begin{corollary}
  Para cualquier gráfica $G$, se tiene que $\bar{d} = 2 e(G) /n.$
\end{corollary}

\begin{definition}
  Una gráfica $K_n$ en $n$ vértices es \textit{completa} si cualquier pareja
  de vértices son adyacentes.
\end{definition}

\begin{definition}
  Una gráfica es \textit{$k$-partita} si sus vértices pueden ser divididos
  en $k$ \textit{partes} de tal manera que todas sus aristas tienen extremos en
  partes distintas. Cuando $k=2$, es decir, cuando solo se tienen dos partes,
  decimos que la gráfica es \textit{bipartita}.
  Frecuentemente, denotaremos a una gráfica bipartia $H$
  con partes $A$ y $B$ como $G[A, B]$. Decimos que una gráfica
  $k$-partita es \textit{completa} si cualquier pareja de vértices en
  partes distintas son adyacentes.
\end{definition}

\begin{definition}
  Denotemos por $K_{s, t}$ a la gráfica bipartita completa con partes
  de cardinalidad $s$ y $t$. La \textit{gráfica de Turán} $T_{k, n}$ es
  una gráfica $k$-partita completa en $n$ vértices cuyas partes
  difieren en cardinalidad a lo más en un vértice.
\end{definition}

\begin{definition}[clique]
  Un clique en una gráfica $G$ es un subconjunto de vértices en el
  cual todos sus vértices son adyacentes.
\end{definition}

\begin{definition}[conjunto independiente]
  Un conjunto independiente en una gráfica $G$ es un subconjunto de
  vértices en el cual ninguno de sus vértices es adyacente.
\end{definition}

\begin{definition}
  Para dos funciones $f$ y $g$, excribimos $f = O(g)$, si $f \leq c g$
  para alguna constante positiva $c$ y para valores suficientemente grandes de
  las variables de $f$ y $g$. Si la desigualdad anterior es estricta,
  esto es, si $f < cg$, entonces escribimos $f = o(g)$.
\end{definition}

% \chapter{Teoría Extremal}

\section{La Subgráfica Prohibida}
Dada una gráfica $H$, el \textit{número de Turán} o \textit{extremal}
$ex(n, H)$, representa el número máximo de aristas de una gráfica en
$n$ vértices que no contiene a $H$, \textit{subgráfica prohibida},
como subgráfica.

\section{Teorema de Turán}
Una gráfica \textit{$k-$ partita} es aquella cuyo conjunto de
vértices puede ser particionado en $k$ subconjuntos, o
\textit{partes}, de tal manera que ninguna arista tiene como extremos
a vértices de una misma parte. Una gráfica $k-$partita es
\textit{completa} si cualesquiera dos vértices en diferentes partes
son adyacentes. Una gráfica completa $k$-partita en $n$ vértices
cuyas partes tienen tamaños que difieren a lo más en un vértice se
les conoce como  \textit{gráficas de Turán} y se denotan como $T_{k, n}$.

\begin{theorem} \label{turan}
  Sea $G$ una gráfica que no contiene a $K_k$, para $k \geq
  2$. Entonces $e(G) \leq e(T_{{k-1}, n})$, con igualdad si y solo si
  $G \simeq T_{{k-1}, n}$.
\end{theorem}

% TODO: agregar cita de Zykov (1949)
% NOTE: demostración del Bondy-Murty
\begin{proof}
  Procederemos por inducción sobre $k$. El teorema se satisface
  trivialmente para $k=2$.  Suponga que se satisface para cualquier
  entero menor que $k$, y sea $G$ una gráfica simple que no contiene
  a $K_k$. Seleccione un vértice $v$ con grado máximo $\Delta$ en
  $G$. Haga $A = N(v)$ y $B = V \setminus A$. Entonces
  $$e(G) = e(A) + e(A, B) + e(B).$$
  Dado que $G$ no contiene a ningún $K_k$, la gráfica inducida $G[A]$
  no contiene a $K_{k-1}$. Por tanto, por inducción,
  $$e(A) \leq e(T_{k-2, \Delta}),$$
  con igualdad si y solo si $G[A] \simeq T_{{k-2}, \Delta}$. Además,
  dado que toda arista de $G$ incidente a un vértice de $B$ pertence
  a $E[A, B]$ o $E( B )$,
  $$e(A, B) + e(B) \leq \Delta (n - \Delta),$$
  con igualdad si y solo si $B$ es un conjunto independiente y cuyos
  miembros tienen grado $\Delta$. Luego, $e(G) \leq e(H)$, donde $H$
  es una gráfica obtenida de una copia de $T_{{k-2}, \Delta}$ y
  añadiendo una copia de un conjunto independiente de $n-\Delta$
  vértices y uniendo cada uno de estos vértices con cada vértice de
  $T_{{k-2}, \Delta}$. Observe que $H$ es una gráfica completa
  $(k-1)$-partita en $n$ vértices. Entonces $e(H) \leq e(T_{{k-1},
  n})$ con igualdad si y solo si $H \simeq T_{{k-1}, n}$. Se sigue
  que $e(G) \leq e(T_{k-1, n})$, con igualdad si y solo si $G \simeq
  T_{k-1, n}$.
\end{proof}
% TODO: agregar demostracion usando probabilistic method

Sea $G$ una gráfica. Recuerde que un conjunto independiente $S$ de
vértices en $G$ es aquel en el que ninguna arista en $G$ tiene como
extremos dos vértices en $S$. De manera complementaria, un clique $C$
en $G$ es un conjunto de vértices en $G$ en el cual cualquier par de
vértices es adyacente. Es decir, un clique $C$ con $\vert C \vert =
k$ induce una gráfica isomorfa a $K_k$, $G[C] \equiv K_k$. Por tanto
el Teorema de Turán puede reescribirse como \textit{toda gráfica $G$
  en $n$ vértices y con más de $e(T_{k-1, n})$ aristas contiene un
clique de orden $k$}.

Los conjuntos independientes y los cliques están relacionados de la siguiente
manera. Para una gráfica $G$, sea $\alpha(G)$ la cardinalidad del conjunto
independiente de vértices más grande en $G$. Y sea $\omega(G)$ la
cardinaliad del
clique más grande en $G$. Entonces
$$\omega(G) = \alpha(\bar{G}),$$
donde $\bar{G}$ es la gráfica complemento de $G$. Por tanto, toda
proposición cierta para cliques puede reescribirse en términos de
conjuntos independientes. Tal es el caso del Teorema de Turán. El
siguiente resultado fue demostrado por
Caro \cite{caro1979} y Wei  %TODO: agregar la cita de wei

\begin{theorem} (Caro y Wei)
  Sea $G = (V, E)$ una gráfica. Entonces $$ \alpha(G) \geq \sum_{v
  \in V} \frac{1}{d(v) + 1}.$$
\end{theorem}

%TODO: cambiar simbolo de orden
\begin{proof}
  Sea $<$ una orden total de $V$ elegido uniformemente. Defina
  $$I = \{v \in V: vw \in E \Rightarrow v < w \}.$$
  Sea $X_{v}$ la variable aleatoria indicadora para $v \in I$ y $X =
  \sum_{v \in V} X_v = \vert I \vert$. Para cada $v$,
  $$E[X_v] = Pr[v \in I] = \frac{1}{d(v) + 1},$$
  dado que $v \in I$ si y solo si $v$ es el elemento menor entre $v$
  y sus vecinos. Luego
  $$ E[X] = \sum_{v \in V} \frac{1}{d(v) + 1}. $$
  Por tanto existe un orden total $<$ con
  $$\vert I \vert \ge \sum_{v \in V} \frac{1}{d(v)  +1 }.$$
  Pero si $x, y \in I$ y $xy \in E$ entonces $x < y$ y $y < x$, una
  contradicción. Por tanto $I$ es un conjunto independiente y
  $\alpha(G) \ge \vert I \vert.$
\end{proof}

% TODO: revisar el statement del teorema
\begin{theorem} (Turán)
  Sea $G$ una gráfica en $n$ vértices y $e(G) \le e(T_{k-1, n})$
  aristas. Entonces $\alpha(G) \geq k$ con $\alpha(G) = k$ si y solo
  si $G \equiv T_{k-1, n}$.
\end{theorem}

\begin{proof} % TODO: completar
    La gráfica $T_{k-1, n}$ tiene $\sum_{v \in V}(d(v) + 1)^{-1} = $
\end{proof}
\section{Teorema de Erdős-Stone} %TODO: no se si esto lo vaya a incluir

\section{Números de Ramsey}
Un \textit{clique} de una gráfica es un conjunto de vértices todos de
ellos adyacentes.



\chapter{El Método Probabilístico}

\section{Gráficas Aleatorias}

\section{El método básico}

\section{La linealidad del valor esperado}

\section{Selección Dependiente Aleatoria}

\begin{theorem}(Lema Básico)\label{drc}
  Sean $d, m, n, r, u$ enteros positivos. Sea $G$ una gráfica en
  $n$ vértices y grado promedio $d$. Si existe
  entero positivo $t$
  tal que
  $$\frac{d^t}{n^{t-1}} - \binom{n}{r} \left(\frac{m}{n}\right)^t \geq u$$
  entonces $G$ contiene un conjunto $U \subset V(G)$ de vértices tal que
  $\vert U \vert \geq u$ y tal que todo subconjunto $S$ de $U$ con
  $\vert S \vert = r$
  tiene al menos $m$ vecinos en común.
\end{theorem}


\section{Números de Turán para Gráficas Bipartitas}

\begin{theorem}\label{turan-drc}
  Si $H = (A \cup B, F)$ es una gráfica bipartita en la que todos los
  vértices de $B$ tienen grado
  a lo más $r$, entonces $ex(n, H) \leq c n^{2 - 1 / r}$, donde $c =
  c(H)$ es una constante que solo depende de $H$.
\end{theorem}
\begin{proof}
  Sean $\vert A \vert = a$, $\vert B
  \vert = b$, $m = a + b$, $t = r$
  y $c = \max(a^{1/r}, \frac{3(a + b)}{r})$. Sea $G$ una gráfica con
  $e(G) > 2 c n^{2 -1 /
  r}$. De manera que $d = 2e(G) / n > 2 c n^{1 - 1/r}$. Entonces

  \begin{eqnarray*}
    \frac{d^{t}}{n^{t-1}} - \binom{n}{r} \left(\frac{m}{n}\right)^t
    &=& \frac{(2c
    n^{1-1/r})^r}{n^{r-1}}  - \binom{n}{r} \left(\frac{a + b}{n}\right)^r\\
    &\geq& (2c)^r - \frac{n^r}{r!} \left((\frac{a + b}{n}\right)^r \\
      &\geq& (2c)^r - \left(\frac{e (a + b)}{r}\right)^r\\
      &\geq& c^r \geq a
    \end{eqnarray*}
    Por tanto, el Teorema \ref{drc} establece que existe un subconjunto
    $U$ de $V(G)$ tal que $\vert U \vert = a$ y en el que todos sus
    subconjuntos de tamaño
    $r$ tienen al menos $a + b$ vecinos en común.

    Sea $f: A \rightarrow U$ una función inyectiva cualquiera.
    Demostraremos que $f$ se puede extender a $B$ de tal manera que
    esta extensión es un encaje de $H$ en $G$. Etiquete los vértices de $B$ como
    $v_1, \ldots, v_b$. Suponga que el vértice por encajar es $v_i \in
    B$. Sea $N_i \subset A$ la vecindad de $v_i$ en $H$; por lo que
    $\vert N_i \vert \leq r$. Dado que $f(N_i)$ es un subconjunto de
    $U$ de cardinalidad a lo más $r$, existen al menos $a + b$ vértices
    adyacentes a todos los vértices en $f(N_i)$. Como el total de
    vértices por encajar es $a + b$, existe un vértice $v \in V(G)$ que
    no se ha usado en el encaje y que es adyacente a $f(N_i)$ en $G$.
    Haga $f(v_i)  = v$. Se observa que este procedimiento puede
    continuar hasta terminar de definir $f(v_i)$ para todo $i = 1,
    \ldots, b$. Por tanto $f$ es un encaje de  $H$ en $G$.
  \end{proof}

  \section{Números de Ramsey para Cubos}
  Para una gráfica $H$, el \textit{número de Ramsey} $r(H)$, es el
  mínimo entero positivo $N$ tal que cualquier \textit{2-coloración}
  de las aristas de la gráfica completa en $N$ vértices contiene una
  copia monocrómatica de $H$.
  \begin{theorem}
    $r(Q_r) \leq 2^{3r}$.
  \end{theorem}
  \begin{proof}
    En cualquier 2-coloración de las aristas de una gráfica comleta
    en $N = 2^{3r}$ vértices, el conjunto más denso entre los dos
    colores tiene al menos
    $\frac{1}{2} \binom{N}{2} \geq 2^{-7/3} N^2$ aristas. Sea $G$ la gráfica
    del color más denso. De manera que el grado promedio $d$ de $G$
    es al menos $2^{-4/3}N$. Aplicando el Teorema \ref{drc} con $t =
    \frac{3}{2}r$, $m = 2^r$ y $a = 2^{r-1}$, tenemos que
    $$\frac{d^t}{N^{t-1}} - \binom{N}{r} \left(\frac{m}{N}\right)^t
    \geq 2^{-4/3 t} N - N^{r-t}m^t / r! \geq 2^r - 1 \geq 2^{r-1}. $$
    Por tanto, existe un subconjunto $U$ de $G$ de tamaño $2^{r-1}$
    en el que todo $r$ vértices en $U$ tiene al menos $2^r$ vecinos en común.
    Dado que $Q_r$ es una gráfica bipartita $r$-regular con partes de
    tamaño $2^{r-1}$, podemos encontrar un encaje de $Q_r$ en $G$ tal
    y como lo hicimos en el Teorema \ref{turan-drc}
  \end{proof}
  % \section{Un Problema del tipo Turán-Ramsey para Gráficas libres de $K_4$}

% \chapter{Comparación de Selección Dependiente Aleatoria con otros Métodos}

\section{Gráficas Bipartitas}


\end{document}


% TODO:
% En Introduccion
% 1. reescribir conceptos de gráfica (no me gusta la función de incidencia)
% 2. Demostrar que la gráficas de Turán son las que maximizan número de aristas
% 3. Demostrar el lema del saludo (sin matrices)

% TODO:
% En Metodo probabilistico
% 1. Introducir concepto de variable aleatoria, espacio muestral, valor esperado... linealidad
% 2. Demostrar versión alternativa de DRC

% TODO:
% 1. Escribur sobre la subgráfica prohibida
% 2. Reescribir teorema de turan
% 3. Introducir números de Ramsey
% 4. Poner demostración de número extremal para gráficas biparitas (original)
% 5. Poner sección de números de Ramsey para cubos


