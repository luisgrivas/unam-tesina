

\chapter{El Método Probabilístico}

\section{El método básico}

\section{La linealidad del valor esperado}

\section{Selección Dependiente Aleatoria}
Para un vértice $v$ en una gráfica $G$, sea $N(v)$ el conjunto de
vecinos de $v$ en $G$. Dado un subconjunto de vértices $U \subset
V(G)$, la \textit{vecindad común} $N(U)$ de $U$ en $G$ es el conjunto
de todos los vértices en $G$ que son adyacentes a todo vértice de $U$.
Un $d-$conjunto es un conjunto de cardinalidad $d$. Para $A$
conjunto, denotaremos a $\binom{A}{d}$ como la familia de
$d-$subconjuntos de $A$.

\begin{theorem}
  Sea $t$ un entero positivo. Sea $G$ una gráfica en $n$ vértices con
  grado promedio $d = \epsilon n$.
  Entonces $G$ continee un conjunto de $t$ vértices cuya vecindad
  tiene al menos $\epsilon^t n - \binom{t}{2}$ vecinos en común.
\end{theorem}
\begin{proof}
    Sean $v_1, \ldots, v_t$ vértices en $V(G)$.

\end{proof}

\begin{theorem}(Lema Básico)\label{drc}
  Sean $d, m, n, r, u$ enteros positivos. Sea $G$ una gráfica en
  $n$ vértices y grado promedio $d$. Si existe
  entero positivo $t$
  tal que
  $$\frac{d^t}{n^{t-1}} - \binom{n}{r} \left(\frac{m}{n}\right)^t \geq u$$
  entonces $G$ contiene un conjunto $U \subset V(G)$ de vértices tal que
  $\vert U \vert \geq u$ y tal que todo subconjunto $S$ de $U$ con
  $\vert S \vert = r$
  tiene al menos $m$ vecinos en común.
\end{theorem}
\begin{proof}
  Seleccione $t$ vértices $v_1, \ldots, v_t$ elegidos uniformemente
  con repetición y defina $T = \{v_1, \ldots, v_t\}$. Sea $A = N(T)$,
  la vecindad de $T$ en $G$. ¿Cuál es la cardinalidad de $A$? Por
  linealidad del valor esperado, se tiene que
  \begin{eqnarray*}
    E[\vert A \vert] &=& \sum_{v \in V(G)} Pr(v \in A)\\
    &=& \sum_{v \in V(G)} \left(\frac{\vert N(v) \vert}{n} \right)^t\\
    &=& n^{-t} \sum_{v \in V(G)} \vert N(v) \vert^t \\
    &\ge& n^{1-t}\left( \frac{\sum_{v \in V(G)} \vert N(v) \vert}{n} \right)^t\\
    &=& \frac{d^t}{n^{t-1}},
  \end{eqnarray*}
  donde la desigualdad se debe a la convexidad de la función $z \mapsto z^t$.

  Sea $Y$ la variable aleatoria que cuenta
  el número de conjuntos $S$ en $\binom{A}{r}$ que tienen menos de
  $m$ vecinos en común. Note que la probabilidad de que un
  $r-$conjunto de vértices cualquiera sea subconjunto de $A$ es
  $\left( \frac{\vert N(S) \vert}{n}\right)^t$. A lo más existen
  $\binom{n}{r}$ subconjuntos $r-$subconjuntos con $\vert N(S)\vert < m$. Luego,
  $$ E[Y] < \binom{n}{r}\left(\frac{m}{n}\right)^t. $$
  Por linealidad del valor esperado,
  $$E[X - Y] \geq \frac{d^t}{n^{t-1}} - \binom{n}{r}\left(\frac{m}{n}
  \right)^t  \ge u.$$
  Por tanto existe una elección de $T$ en el cual el conjunto $A =
  N(T)$ satisface que $X - Y \ge u$.
  Sea $U$ el conjunto resultante al borrar un vértice de cada
  $r$-subconjunto $S$ de $A$ que tenga menos de $m$ vecinos en común.
  Entonces $\vert U \vert \ge u$ y todos los $r-$subconjuntos de $U$
  tienen al menos $m$ vecinos en común.
\end{proof}

\begin{theorem}
  Si $\epsilon > 0$ y $d\leq n$ son enteros positivos, y $G = (V, E)$
  es una gráfica con
  $N > 4d\epsilon^{-d}n$ vértices y al menos $\epsilon N^2 / 2$
  aristas, entonces existe un
  subconjunto $U$ de vértices de $G$ con $\vert U \vert > 2n$ tal que
  la fracción de $d-$conjuntos $S \subset U$ con $\vert N(S) < n$ es
  menor que $(2d)^{-d}$.
\end{theorem}
