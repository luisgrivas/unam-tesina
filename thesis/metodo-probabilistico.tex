% \chapter{El Método Probabilístico}
\section{El Método Probabilístico}
El \textit{método probabilista} es una técnica importante en
combinatoria. Con frecuencia, deseamos probar la existencia de
ciertra propiedad u objeto. Para ello, el método probabilista construye
de manera aleatoria este objeto con probabilidad mayor que cero.

\begin{theorem}
  Toda gráfica con $m$ vértices contiene una gráfica bipartita con $m
  / 2$ aristas.
\end{theorem}

\begin{proof}
  Sea $G = (V, E)$ una gráfica. Asigne un color, blanco o negro, a
  cada vértice de $G$  de manera uniforme e independiente.

  Sea $E^\prime$ el conjunto de aristas con un extremo negro y otro
  blanco. La gráfica $H = (V, E^\prime)$ es una subgráfica bipartita de $G$.
  Toda arista de $G$ pertenece a $E^\prime$ con probabilidad $1/2$.
  Por linealidad del valor esperado, el tamaño esperado de $E^\prime$ es
  $$\E[\vert E^\prime \vert] = \frac{1}{2} \vert E \vert.$$
  Entonces existe una coloración de tal manera que $\vert E^\prime
  \vert \ge \frac{1}{2}\vert E \vert$. La gráfica $H = (V, E^\prime)$
  es la subgráfica deseada.
\end{proof}

\section{La linealidad del valor esperado}
Sean $X_1, \ldots, X_n$ variables aleatorias y sean $X = c1 X_1 + \ldots + c_n X_n$, donde $c_i$ son números reales. La \textit{linealidad del valor esperado} establece que 
$$ \E[X] = c_1 \E[X_1] + \ldots + c_n \E[X_n]. $$
\section{Desigualdad de Jensen}

\section{Zerekiewiecz}

\section{Kovari-Sos-Turán}

\begin{theorem}[Kovari-Sos-Turán]
  Sean $2 \leq s \leq t$ y sea $K_{s,t}$ la gráfica bipartita
  completa con $s + t$ vértices. Entonces existe una constante $c =
  c(s, t) > 0$ tal que
  la gráfica en $n$ vértices que no contiene una copia de $K_{s, t}$
  tiene a lo más
  $c n^{2 - 1/s}$ aristas.
\end{theorem}
\begin{proof}
  Sea $G$ una gráfica con a lo más $cn^{2 - 1/s}$ aristas.
  Sea $\{u_1, \ldots, u_t\}$ un subconjunto de $t$ vértices de $G$.

  Etiquete
  los vértices de $G$ como $v_1, \ldots, v_n$. Para cada vértice
  $v_i$, existen exactamente $\binom{d(v_i)}{t}$ conjuntos $\{v_1,
  \ldots, v_t\}$
\end{proof}

\section{Selección Dependiente Aleatoria}
Para un vértice $v$ en una gráfica $G$, sea $N(v)$ el conjunto de
vecinos de $v$ en $G$. Dado un subconjunto de vértices $U \subset
V(G)$, la \textit{vecindad común} $N(U)$ de $U$ en $G$ es el conjunto
de todos los vértices en $G$ que son adyacentes a todo vértice de $U$.
Un $d-$conjunto es un conjunto de cardinalidad $d$. Para $A$
conjunto, denotaremos a $\binom{A}{d}$ como la familia de
$d-$subconjuntos de $A$.

\begin{theorem}
  Sea $t$ un entero positivo. Sea $G$ una gráfica en $n$ vértices con
  grado promedio $d = \epsilon n$.
  Entonces $G$ continee un conjunto de $t$ vértices cuya vecindad
  tiene al menos $\epsilon^t n - \binom{t}{2}$ vecinos en común.
\end{theorem}
\begin{proof} %TODO: proof
  Sean $v_1, \ldots, v_t$ vértices seleccionados uniformemente en
  $V(G)$ y sea $T = \{v_1, \ldots, v_t\}$. Para un vértice $u$, la
  probabilidad que $u$ es una vecino en común de $T$ es
  $\left(\frac{d(v)}{n}\right)^t$.
  Por tanto, si $A = N(T)$, se tiene que
  \begin{eqnarray*}
    E[\vert A \vert] &=& \sum_{v \in V(G)} Pr(v \in A) \\
    &=& \sum_{v \in V(G)} \left(\frac{d(v)}{n} \right)^t \\
    &\ge& n \left(\frac{d}{n}\right)^t\\
    &=& \epsilon^t n,
  \end{eqnarray*}
  donde la desigualdad se sigue de la convexidad de la función $z \mapsto z^t$.
  Por tanto, existe una elección del conjunto $T$ con al menos $\epsilon^t n $.

  Note que es posible que en la elección de $T$ existan vértices
  repetidos, esto es, $\vert T \vert < t$. Note que $Pr(v_i = v_j) =
  1 / n$, entonces
  $Pr(\vert T \vert < t) \le \binom{t}{2} / n$. Dado que todo
  vértices $v$ satisface que
  $d(v) < n$, entonces
  \begin{eqnarray*}
    \epsilon^t n &\le& E[\vert A \vert] \\
    &=& E[\vert A \vert | \vert T \vert = t] Pr(\vert T \vert = t) +
    E[\vert A \vert | \vert T \vert < t] Pr(\vert T \vert < t)  \\
    &\le& E[\vert A \vert | \vert T \vert = t] + \binom{t}{2}
  \end{eqnarray*}
\end{proof}

\begin{theorem}(Lema Básico)\label{drc}
  Sean $d, m, n, r, u$ enteros positivos. Sea $G$ una gráfica en
  $n$ vértices y grado promedio $d$. Si existe
  entero positivo $t$
  tal que
  $$\frac{d^t}{n^{t-1}} - \binom{n}{r} \left(\frac{m}{n}\right)^t \geq u$$
  entonces $G$ contiene un conjunto $U \subset V(G)$ de vértices tal que
  $\vert U \vert \geq u$ y tal que todo subconjunto $S$ de $U$ con
  $\vert S \vert = r$
  tiene al menos $m$ vecinos en común.
\end{theorem}
\begin{proof}
  Seleccione $t$ vértices $v_1, \ldots, v_t$ elegidos uniformemente
  con repetición y defina $T = \{v_1, \ldots, v_t\}$. Sea $A = N(T)$,
  la vecindad de $T$ en $G$. ¿Cuál es la cardinalidad de $A$? Por
  linealidad del valor esperado, se tiene que
  \begin{eqnarray*}
    E[\vert A \vert] &=& \sum_{v \in V(G)} Pr(v \in A)\\
    &=& \sum_{v \in V(G)} \left(\frac{\vert N(v) \vert}{n} \right)^t\\
    &=& n^{-t} \sum_{v \in V(G)} \vert N(v) \vert^t \\
    &\ge& n^{1-t}\left( \frac{\sum_{v \in V(G)} \vert N(v) \vert}{n} \right)^t\\
    &=& \frac{d^t}{n^{t-1}},
  \end{eqnarray*}
  donde la desigualdad se debe a la convexidad de la función $z \mapsto z^t$.

  Sea $Y$ la variable aleatoria que cuenta
  el número de conjuntos $S$ en $\binom{A}{r}$ que tienen menos de
  $m$ vecinos en común. Note que la probabilidad de que un
  $r-$conjunto de vértices cualquiera sea subconjunto de $A$ es
  $\left( \frac{\vert N(S) \vert}{n}\right)^t$. A lo más existen
  $\binom{n}{r}$ subconjuntos $r-$subconjuntos con $\vert N(S)\vert < m$. Luego,
  $$ E[Y] < \binom{n}{r}\left(\frac{m}{n}\right)^t. $$
  Por linealidad del valor esperado,
  $$E[X - Y] \geq \frac{d^t}{n^{t-1}} - \binom{n}{r}\left(\frac{m}{n}
  \right)^t  \ge u.$$
  Por tanto existe una elección de $T$ en el cual el conjunto $A =
  N(T)$ satisface que $X - Y \ge u$.
  Sea $U$ el conjunto resultante al borrar un vértice de cada
  $r$-subconjunto $S$ de $A$ que tenga menos de $m$ vecinos en común.
  Entonces $\vert U \vert \ge u$ y todos los $r-$subconjuntos de $U$
  tienen al menos $m$ vecinos en común.
\end{proof}

\begin{theorem}
  Si $\epsilon > 0$ y $d\leq n$ son enteros positivos, y $G = (V, E)$
  es una gráfica con
  $N > 4d\epsilon^{-d}n$ vértices y al menos $\epsilon N^2 / 2$
  aristas, entonces existe un
  subconjunto $U$ de vértices de $G$ con $\vert U \vert > 2n$ tal que
  la fracción de $d-$conjuntos $S \subset U$ con $\vert N(S) < n$ es
  menor que $(2d)^{-d}$.
\end{theorem}
