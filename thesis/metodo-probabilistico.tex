% \chapter{El Método Probabilístico}
\section{El Método Probabilístico}
El \textit{método probabilista} es una técnica importante en
combinatoria para probar la existencia de cierta propiedad u objeto, pero en el
que un método constructivo para hayarlo es complicado de idear.

\begin{definition}[$\sigma$-álgebra]
  Sea $\Omega$ un conjunto. Una $\sigma$-álgebra $\Sigma$ en $\Omega$ es una
  familia de subconjuntos de $\Omega$ que satisface que
  \begin{itemize}
    \item $\varnothing$ y $\Omega $ son elementos de $\Sigma$.\\
    \item Si $A \in \Sigma$, entonces $\Sigma \setminus A \in \Sigma$.\\
    \item Si $\{A_n\}_{n\in \N}$ es una sucesión de elementos en
      $\Sigma$, entonces
      $$\bigcup_{n \in \N} A_n \in \Sigma.$$
  \end{itemize}
\end{definition}

\begin{definition}[medida]
  Una medida $\mu \Sigma: \rightarrow [0, \infty]$ es una función tal que
  \begin{itemize}
    \item $\mu(\varnothing) = 0$.\\
    \item Para $\{A_n\}_{n\in \N}$ en $\Sigma$, con $A_m \cap A_n =
      \varnothing$ si $m \neq n$, se tiene que
      $$ \mu\left( \bigcup_{n \in \N} A_n \right) = \sum_{n \in \N} \mu(A_n).$$
  \end{itemize}
\end{definition}

\begin{definition}[Espacio de Probabilidad]
  Un espacio de probabilidad es una tripleta $(\Omega, \Sigma, Pr)$,
  donde $\Sigma$ es una $\sigma$-álgebra en $\Omega$, $Pr$ es una
  medida en $\Omega$ tal que $Pr(\Omega) = 1$. A los elementos de
  $\Sigma$ les llamamos eventos.
\end{definition}

\begin{definition}[Variable Aleatoria]
  Una variable aleatoria real en un espacio de probabilidad $(\Omega,
  \Sigma, Pr)$ es una
  función $X: \Omega \rightarrow \mathbb R$ $Pr$-medible. Esto es,
  para todo $\alpha \in \mathbb R$
$$\{\omega \in \mathbb R: X(\omega) \le \alpha \} \in \Sigma \}.$$
\end{definition}

\begin{definition}[Valor Esperado]
El valor esperado de una variable aleatoria $X$ se define como
$$\E[X] = \int_{\Omega} X(\omega) d Pr(\omega).$$
\end{definition}

Recuerde que la integral (de Lebesgue) satisface las propiedades de
linealidad. Por lo que, el valor esperado satisface el siguiente Teorema.

\begin{theorem}
Sean $X_1, \ldots, X_n$ variables aleatorias y sean $X = c1 X_1 +
\ldots + c_n X_n$, donde $c_i$ son números reales. La
\textit{linealidad del valor esperado} establece que
$$ \E[X] = c_1 \E[X_1] + \ldots + c_n \E[X_n]. $$
\end{theorem}

\begin{theorem}
Toda gráfica con $m$ vértices contiene una gráfica bipartita con $m
/ 2$ aristas.
\end{theorem}
\begin{proof}
Sea $G = (V, E)$ una gráfica. Asigne un color, blanco o negro, a
cada vértice de $G$  de manera uniforme e independiente.

Sea $E^\prime$ el conjunto de aristas con un extremo negro y otro
blanco. La gráfica $H = (V, E^\prime)$ es una subgráfica bipartita de $G$.
Toda arista de $G$ pertenece a $E^\prime$ con probabilidad $1/2$.
Por linealidad del valor esperado, el tamaño esperado de $E^\prime$ es
$$\E[\vert E^\prime \vert] = \frac{1}{2} \vert E \vert.$$
Entonces existe una coloración de tal manera que $\vert E^\prime
\vert \ge \frac{1}{2}\vert E \vert$. La gráfica $H = (V, E^\prime)$
es la subgráfica deseada.
\end{proof}

% \section{Desigualdad de Jensen}
%
% \section{Zerekiewiecz}

\subsection{Kovari-Sos-Turán}
\begin{theorem}[Kovari-Sos-Turán]
Sean $2 \leq s \leq t$ y sea $K_{s,t}$ la gráfica bipartita
completa con $s + t$ vértices. Entonces existe una constante $c =
c(s, t) > 0$ tal que
la gráfica en $n$ vértices que no contiene una copia de $K_{s, t}$
tiene a lo más
$c n^{2 - 1/s}$ aristas.
\end{theorem}
\begin{proof}
Sea $G$ una gráfica con a lo más $cn^{2 - 1/s}$ aristas.
Sea $\{u_1, \ldots, u_t\}$ un subconjunto de $t$ vértices de $G$.
Etiquete los vértices de $G$ como $v_1, \ldots, v_n$. Para cada vértice
$v_i$, existen exactamente $\binom{d(v_i)}{t}$ conjuntos $\{v_1,
\ldots, v_t\}$
\end{proof}

\subsection{Selección Dependiente Aleatoria}
% Para un vértice $v$ en una gráfica $G$, sea $N(v)$ el conjunto de
% vecinos de $v$ en $G$. Dado un subconjunto de vértices $U \subset
% V(G)$, la \textit{vecindad común} $N(U)$ de $U$ en $G$ es el conjunto
% de todos los vértices en $G$ que son adyacentes a todo vértice de $U$.

Un $d-$conjunto es un conjunto de cardinalidad $d$. Para $A$
conjunto, denotaremos a $\binom{A}{d}$ como la familia de
$d-$subconjuntos de $A$.

\begin{theorem}
Sea $t$ un entero positivo. Sea $G$ una gráfica en $n$ vértices con
grado promedio $d = \epsilon n$.
Entonces $G$ continee un conjunto de $t$ vértices cuya vecindad
tiene al menos $\epsilon^t n - \binom{t}{2}$ vecinos en común.
\end{theorem}
\begin{proof} %TODO: proof
Sean $v_1, \ldots, v_t$ vértices seleccionados uniformemente en
$V(G)$ y sea $T = \{v_1, \ldots, v_t\}$. Para un vértice $u$, la
probabilidad que $u$ es una vecino en común de $T$ es
$\left(\frac{d(v)}{n}\right)^t$.
Por tanto, si $A = N(T)$, se tiene que
\begin{eqnarray*}
  E[\vert A \vert] &=& \sum_{v \in V(G)} Pr(v \in A) \\
  &=& \sum_{v \in V(G)} \left(\frac{d(v)}{n} \right)^t \\
  &\ge& n \left(\frac{d}{n}\right)^t\\
  &=& \epsilon^t n,
\end{eqnarray*}
donde la desigualdad se sigue de la convexidad de la función $z \mapsto z^t$.
Por tanto, existe una elección del conjunto $T$ con al menos $\epsilon^t n $.

Note que es posible que en la elección de $T$ existan vértices
repetidos, esto es, $\vert T \vert < t$. Note que $Pr(v_i = v_j) =
1 / n$, entonces
$Pr(\vert T \vert < t) \le \binom{t}{2} / n$. Dado que todo
vértices $v$ satisface que
$d(v) < n$, entonces
\begin{eqnarray*}
  \epsilon^t n &\le& E[\vert A \vert] \\
  &=& E[\vert A \vert | \vert T \vert = t] Pr(\vert T \vert = t) +
  E[\vert A \vert | \vert T \vert < t] Pr(\vert T \vert < t)  \\
  &\le& E[\vert A \vert | \vert T \vert = t] + \binom{t}{2}
\end{eqnarray*}
\end{proof}

\begin{theorem}(Lema Básico)\label{drc}
Sean $d, m, n, r, u$ enteros positivos. Sea $G$ una gráfica en
$n$ vértices y grado promedio $d$. Si existe
entero positivo $t$
tal que
$$\frac{d^t}{n^{t-1}} - \binom{n}{r} \left(\frac{m}{n}\right)^t \geq u$$
entonces $G$ contiene un conjunto $U \subset V(G)$ de vértices tal que
$\vert U \vert \geq u$ y tal que todo subconjunto $S$ de $U$ con
$\vert S \vert = r$
tiene al menos $m$ vecinos en común.
\end{theorem}
\begin{proof}
Seleccione $t$ vértices $v_1, \ldots, v_t$ elegidos uniformemente
con repetición y defina $T = \{v_1, \ldots, v_t\}$. Sea $A = N(T)$,
la vecindad de $T$ en $G$. ¿Cuál es la cardinalidad de $A$? Por
linealidad del valor esperado, se tiene que
\begin{eqnarray*}
  E[\vert A \vert] &=& \sum_{v \in V(G)} Pr(v \in A)\\
  &=& \sum_{v \in V(G)} \left(\frac{\vert N(v) \vert}{n} \right)^t\\
  &=& n^{-t} \sum_{v \in V(G)} \vert N(v) \vert^t \\
  &\ge& n^{1-t}\left( \frac{\sum_{v \in V(G)} \vert N(v) \vert}{n} \right)^t\\
  &=& \frac{d^t}{n^{t-1}},
\end{eqnarray*}
donde la desigualdad se debe a la convexidad de la función $z \mapsto z^t$.

Sea $Y$ la variable aleatoria que cuenta
el número de conjuntos $S$ en $\binom{A}{r}$ que tienen menos de
$m$ vecinos en común. Note que la probabilidad de que un
$r-$conjunto de vértices cualquiera sea subconjunto de $A$ es
$\left( \frac{\vert N(S) \vert}{n}\right)^t$. A lo más existen
$\binom{n}{r}$ subconjuntos $r-$subconjuntos con $\vert N(S)\vert < m$. Luego,
$$ E[Y] < \binom{n}{r}\left(\frac{m}{n}\right)^t. $$
Por linealidad del valor esperado,
$$E[X - Y] \geq \frac{d^t}{n^{t-1}} - \binom{n}{r}\left(\frac{m}{n}
\right)^t  \ge u.$$
Por tanto existe una elección de $T$ en el cual el conjunto $A =
N(T)$ satisface que $X - Y \ge u$.
Sea $U$ el conjunto resultante al borrar un vértice de cada
$r$-subconjunto $S$ de $A$ que tenga menos de $m$ vecinos en común.
Entonces $\vert U \vert \ge u$ y todos los $r-$subconjuntos de $U$
tienen al menos $m$ vecinos en común.
\end{proof}

\begin{theorem}
Si $\epsilon > 0$ y $d\leq n$ son enteros positivos, y $G = (V, E)$
es una gráfica con
$N > 4d\epsilon^{-d}n$ vértices y al menos $\epsilon N^2 / 2$
aristas, entonces existe un
subconjunto $U$ de vértices de $G$ con $\vert U \vert > 2n$ tal que
la fracción de $d-$conjuntos $S \subset U$ con $\vert N(S) < n$ es
menor que $(2d)^{-d}$.
\end{theorem}
\begin{proof}
Sea $T$ un subconjunto de $d$ vértices escogidos uniformemente con
repetición. Sea
$U = N(T)$ y sea $X$ la cardinalidad de $U$. Por linealidad del valor
esperado y por
convexidad de la función $z \mapsto z^d$, se tiene que
\begin{eqnarray*}
  \E[X] &=& \sum_{v \in V}Pr(v \in U)\\
  &=& \sum_{v \in V}\left(\frac{\vert N(v) \vert}{N}\right)^d\\
  &=& N^{-d}\sum_{v \in V} \vert N(V) \vert^d \\
  &\ge& N^{1-d}\sum_{v \in V} \left( \frac{\sum_{v \in V}\vert
  N(v)\vert}{N} \right)^d\\
  &\ge& \epsilon^dN.
\end{eqnarray*}
Sea $Y$ la variable aleatoria que cuenta el número de $d$-conjuntos
en $U$ con menos de
$n$ vecinos en común. Dado un $d$-conjunto $S$, la probabilidad de
que $S$ sea subconjunto de $U$
es $\left(\frac{\vert N(S)\vert}{N} \right)^d$. Por tanto, tenemos que
$$\E[Y] = \binom{N}{d} \left( \frac{n-1}{N} \right)^d.$$

Por convexidad, $\E[X^d] \ge \E[X]^d$. Luego, por linealidad del
valor esperado, obtenemos que
$$\E\left[X^d - \frac{\E[X]^d}{2 \E[Y]} Y - \frac{\E[X]^d}{2} \right] \ge 0.$$

Por tanto, existe una elección de $T$ para el cual la expresión
anterior no es negativa. Entonces
$$X^d \ge \frac{1}{2} \E[X]^d \ge \frac{1}{2}\epsilon^{d^2} N^d, $$
y por tanto $\vert U \vert = X \ge \epsilon^d N / 2 > 2n.$ Además,
\begin{eqnarray*}
  Y &\le& 2X^d\E[Y]\E[X]^{-d} \\
  &<& 2 \vert U \vert^d \binom{N}{d}\left( \frac{n}{N} \right)^d
  \frac{1}{\epsilon^{d^2} N^d}\\
  &<& \left( \frac{2n}{\epsilon^d N} \right)^d \binom{\vert U \vert}{d} \\
  &\le& (2d)^{-d} \binom{\vert U \vert}{d}.
\end{eqnarray*}
donde usamos la desigualdad $\vert U \vert^d \le 2^{d-1} d!
\binom{\vert U \vert}{d}$, que se sigue de $\vert U \vert > 2n \ge 2d$.
\end{proof}
