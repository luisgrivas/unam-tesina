% \chapter{Introducción}

\section{Preliminares}
\begin{definition}
  Una gráfica $G$ es una pareja ordenada $(V, E)$, donde $V$
  es un conjunto cuyos elementos llamamos \textit{vértices} y $E$ es
  un conjunto cuyos elementos son parejas de vértices y que llamamos
  \textit{aristas}.
  %TODO: ver libros de olimpiadas para ver las def en español
\end{definition}

Si $e$ es una arista y $u$ y $v$ son vértices tal que $\psi_G(e)
= \{u, v\}$, entonces se dice que $e$ \textit{une} a los vértices
$u$ y $v$, y a estos vértices se les conoce como \textit{extremos} de $e$.
El grado de un vértice $v$, denotado como $d(v)$, es el número de
aristas en $G$ que inciden en $v$.

\begin{definition}
  Sea $v$ un vértice en una gráfica $G$. Definimos a la vecindad
  $N(v)$ de $v$ como el conjunto de todos los vértices adyacentes a
  $v$ en $G$. De
  manera similar, sSi $U$ es un conjunto, definimos a la vencidad de $U$ como
  el conjunto de vértices en $G$ adyacentes a todo vértice de $U$.
\end{definition}

\begin{definition}
  El grado $d(v)$ de un vértice $v$ en una gráfica $G$ es la
  cardinalidad de la vecindad de $v$, y escribimos
  $d(v) = \vert N(v) \vert$. Al grado máximo en una gráfica $G$ lo
  denotamos como $\Delta$. El \textit{grado promedio} en una gráfica
  se define como $d = \sum_{v \in V} d(v) / n$.
\end{definition}

\begin{theorem}[del Saludo] Para cualquier gráfica $G$ con $m$
  aristas se tiene que
  $$\sum_{v \in V(G)} d(v) = 2m.$$
\end{theorem}
%TODO: proof
El \textit{grado promedio} $\bar{d}$ de una gráfica $G$ en $n$
vértices es $\sum_{ v
\in V(G)} d(v) / n$. Así pues, el Teorema anterior implica que

\begin{corollary}
  Para cualquier gráfica $G$, se tiene que $\bar{d} = 2 e(G) /n.$
\end{corollary}

\begin{definition}
  Una gráfica $K_n$ en $n$ vértices es \textit{completa} si cualquier pareja
  de vértices son adyacentes.
\end{definition}

\begin{definition}
  Una gráfica es \textit{$k$-partita} si sus vértices pueden ser divididos
  en $k$ \textit{partes} de tal manera que todas sus aristas tienen extremos en
  partes distintas. Cuando $k=2$, es decir, cuando solo se tienen dos partes,
  decimos que la gráfica es \textit{bipartita}.
  Frecuentemente, denotaremos a una gráfica bipartia $H$
  con partes $A$ y $B$ como $G[A, B]$. Decimos que una gráfica
  $k$-partita es \textit{completa} si cualquier pareja de vértices en
  partes distintas son adyacentes.
\end{definition}

\begin{definition}
  Denotemos por $K_{s, t}$ a la gráfica bipartita completa con partes
  de cardinalidad $s$ y $t$. La \textit{gráfica de Turán} $T_{k, n}$ es
  una gráfica $k$-partita completa en $n$ vértices cuyas partes
  difieren en cardinalidad a lo más en un vértice.
\end{definition}

\begin{definition}[clique]
  Un clique en una gráfica $G$ es un subconjunto de vértices en el
  cual todos sus vértices son adyacentes.
\end{definition}

\begin{definition}[conjunto independiente]
  Un conjunto independiente en una gráfica $G$ es un subconjunto de
  vértices en el cual ninguno de sus vértices es adyacente.
\end{definition}

\begin{definition}
  Para dos funciones $f$ y $g$, excribimos $f = O(g)$, si $f \leq c g$
  para alguna constante positiva $c$ y para valores suficientemente grandes de
  las variables de $f$ y $g$. Si la desigualdad anterior es estricta,
  esto es, si $f < cg$, entonces escribimos $f = o(g)$.
\end{definition}
