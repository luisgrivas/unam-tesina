% \chapter{Introducción}

\section{Preliminares}
Una gráfica $G$ es una pareja ordenada $(V(G), E(G))$, donde $V(G)$
es un conjunto de \textit{vértices},  $E(G)$ es un conjunto de
\textit{aristas} disjunto de $V(G)$ y una \textit{función de
incidencia} $\psi_G$ que asocia a una arista de $G$ una pareja
de vértices (no necesariamente distintos) de $G$.

Si $e$ es una arista y $u$ y $v$ son vértices tal que $\psi_G(e)
= \{u, v\}$, entonces se dice que $e$ \textit{une} a los vértices
$u$ y $v$, y a estos vértices se les conoce como \textit{extremos} de $e$.

% Denotamos por $v(G)$ y $e(G)$ la cardinalidad de $V(G)$ y $E(G)$
% respectivamente.

El grado de un vértice $v$, denotado como $d(v)$, es el número de
aristas en $G$ que inciden en $v$.

\begin{theorem}[del saludo] Para cualquier gráfica $G$ con $m$
  aristas se tiene que
  $$\sum_{v \in V(G)} d(v) = 2m.$$
\end{theorem}

El \textit{grado promedio} $\bar{d}$ de una gráfica $G$ en $n$
vértices es $\sum_{ v
\in V(G)} d(v) / n$. Así pues, el Teorema anterior implica que

\begin{corollary}
  Para cualquier gráfica $G$, se tiene que $\bar{d} = 2 e(G) /n.$
\end{corollary}

Una gráfica es \textit{bipartita} si sus vértices pueden ser divididos
en dos subconjuntos disjuntos $A$ y $B$, llamados \textit{partes}, de
tal manera que todas las aristas de $G$ tienen un extremo en $A$ y el
otro en $B$. Frecuentemente, denotaremos a una gráfica bipartia $H$
con partes $A$ y $B$ como $G[A, B]$. Si todo vértice de $A$ es adyacente
a todo vértice de $B$, entonces diremos que $G$ es una gráfica
\textit{biparita completa}.
Si la cardinalidad de $A$ es $s$ y la de $B$ es $t$, denotaremos a la
gráfica bipartita completa como $K_{s, t}$.
