\chapter{La desigualdad de Brunn-Minkowski}



\section{Simetrización de Steiner} % Recuerda que esto es una copia y hay que parafrasear


\begin{defi}
    Sea $H$ un hiperplano que intersecta al origen en $\R^3$. Dado un cuerpo convexo $K$ en $\R^3$, el \textit{simétrico de Steiner} de $K$ con respecto a $H$ es un conjunto compacto $S(K)$ con las siguientes propiedades:
    \begin{itemize}
        \item[i] $S(K)$ es simétrico respecto a $H$;
        \item[ii] para toda línea $L$ ortogonal a $H$, $L \cap K = \emptyset$ si y solo si $L \cap S(K) = \emptyset$;
        \item[iii] para toda línea $L$ ortogonal a $H$ que intersecta a $K$, la longitud de $L \cap K $ es exactamente la longitud de $L \cap S(K).$
    \end{itemize}
\end{defi}


\begin{teo}
    \begin{itemize}
        \item[i] $S(K) \subset S(L)$, si $K \subset L.$
        \item[ii] $S(B(o,r)) = B(o,r).$
        \item[iii] $S(\lambda K) = \lambda S(K)$ para toda $\lambda \in \R.$
        \item[iv] Para todo conjunto convexo $K$ en $\R^n$, el simetrico de Steiner $S(K)$ tambien es convexo.
        \item[v] La simetrizacion de Steiner $S: \mathcal{K}^n \rightarrow \matchal{K}^n$ es continua, \textit{volume preserving function.}
    \end{itemize}
\end{teo}
    